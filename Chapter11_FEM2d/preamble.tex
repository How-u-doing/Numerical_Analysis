% !TeX program = pdflatex
\documentclass[a4paper,english]{article}
\usepackage[utf8]{inputenc}
%\usepackage[T1]{fontenc} % to use the accented characters as individual glyphs
						  % automatically hyphenate words for European 
						  % languages (German, French, Italian, etc.)
\usepackage[USenglish]{babel}
\usepackage[none]{hyphenat}
\usepackage[margin=1in]{geometry}
\usepackage{enumerate}
\usepackage{amsmath}
\usepackage{amssymb}
\usepackage{amsfonts}
\usepackage{mathtools}
\DeclarePairedDelimiter\norm{\lVert}{\rVert}
\newcommand{\vertiii}[1]{{\left\vert\kern-0.4ex\left\vert\kern-0.4ex\left\vert
	#1 \right\vert\kern-0.4ex\right\vert\kern-0.4ex\right\vert}}
\usepackage{bm}
\usepackage{cases}
\usepackage{esvect}
\usepackage{amsthm}
\newtheorem{theorem}{Theorem}[section]
\newtheorem{lemma}[theorem]{Lemma}
\newtheorem{corollary}[theorem]{Corollary}
\theoremstyle{definition}
\newtheorem{definition}{Definition}[section]
\theoremstyle{remark}
\newtheorem{remark}{Remark}
\renewcommand\qedsymbol{$\square$}

\usepackage[framemethod=TikZ]{mdframed} % to create elegant boxes, frames
\mdfdefinestyle{mdftheoremstyle}{
	linecolor=green,linewidth=1pt,roundcorner=10pt,
	frametitlerule=true, frametitlerulewidth=.5pt,
	frametitlebackgroundcolor=green!20,
	innertopmargin=\topskip,
}
\mdfdefinestyle{mdfdefinitionstyle}{
	linecolor=blue,linewidth=1pt,roundcorner=10pt,
	frametitlerule=true, frametitlerulewidth=.5pt,
	frametitlebackgroundcolor=blue!20,
	innertopmargin=\topskip,
}
\mdtheorem[style=mdfdefinitionstyle]{mdfdef}{Definition}[section]
\mdtheorem[style=mdftheoremstyle]{mdfthm}{Theorem}[section]
\mdtheorem[style=mdftheoremstyle]{mdflma}{Lemma}[section]
%% NOTE %%
% To be consistent, choose either mdf[def,thm,lma] or the standard ones
% DON'T MIX THEM


\usepackage{standalone}  % to compile alone for efficiency
%\usepackage{afterpage}  % used when adding the cover page
%\usepackage{pdfpages}   % for inclusion of external multi-page PDF
\usepackage{xcolor}
\usepackage{colortbl}

% some user-defined colors
\definecolor{codegray}{rgb}{0.5,0.5,0.5}
\definecolor{background-color}{rgb}{0.95,0.95,0.92}
\definecolor{brilliant-blue}{RGB}{0, 176, 220}
\definecolor{mid-green}{RGB}{0,167,0}


\usepackage{graphicx}
\usepackage{subfig}
\usepackage{wrapfig}
\usepackage[justification=raggedright] % prevents spiltting words with hyphens
			{caption}
%\usepackage{pgfplots} % includes tikz
\usepackage{tikz}
\usetikzlibrary{shapes.geometric,calc,matrix,hobby}
\newcommand*\circled[1]{\tikz[baseline=(char.base)]{
	\node[shape=circle,text=white,fill=black,inner sep=.5pt] (char) {#1};}}	
\newcommand*\orangecircled[1]{\tikz[baseline=(char.base)]{
	\node[shape=circle,fill=orange,inner sep=.5pt] (char) {#1};}}
\newcommand*\greencircled[1]{\tikz[baseline=(char.base)]{
	\node[shape=circle,text=white,fill=mid-green,inner sep=.5pt] (char) {#1};}}
\newcommand\greencircle{\tikz[baseline=(char.base)] 			
	\node[fill=mid-green,circle]{};}
\usepackage{tkz-euclide} % \tkzMarkAngle


\usepackage{algorithm}
%\usepackage{algorithmicx} % automatically loaded by `algpseudocode`
\usepackage{algpseudocode} % layout for algorithmicx

\usepackage[toc,page]{appendix}
%\usepackage[square,numbers]{natbib}
%\bibliographystyle{abbrvnat}

% configuration of biblatex with biber:
%https://tex.stackexchange.com/questions/154751/biblatex-with-biber-configuring-my-editor-to-avoid-undefined-citations
\usepackage[
backend=biber,
style=alphabetic,
sorting=ynt
]{biblatex}
\addbibresource{bibliography.bib}


\usepackage[newfloat]{minted} % compile with -shell-escape flag
% outputdir=./tmp causes problems, since \minted still looks for 
% temporary files in the document root directory.

%\renewcommand{\listingscaption}{Code}% only applies when newfloat is unset
\renewcommand*\listingname{Code listing}

% minted lineno style
\renewcommand{\theFancyVerbLine}{\sffamily
	\textcolor{codegray}{\tiny
		%		\scriptsize% \oldstylenums
		{\arabic{FancyVerbLine}}}}
\setminted{
	mathescape=true,
%	escapeinside=||,
	linenos=true,
	numbersep=1.5mm,
	autogobble,
	breaklines,	breakanywhere,
	fontsize=\footnotesize,
	style=xcode,
	bgcolor=background-color
}
% Create a new environment for breaking code listings across pages.
\newenvironment{code}{\captionsetup{type=listing}}{}


\usepackage{fancyhdr}
\pagestyle{fancy}
\fancyhf{} 	% clears the header and footer of default "plain" page style
\lhead{\leftmark}
\rhead{\thepage}
\chead{\hyperlink{Contents}{Contents}}

\usepackage{hyperref, bookmark}

\hypersetup{
	pdftitle = {FEM2d},
	pdfsubject = {Numerical Analysis},
	pdfkeywords = {Finite Element Methods},
	pdfauthor = {Mark Taylor},
	bookmarks = true,
	bookmarksnumbered = true,
	pdfpagelabels = true,
	pdfpagemode = UseOutlines,
	pdfstartview = FitH,
	linktocpage = true,
	colorlinks = true,
	citecolor = orange, 
	linkcolor = brilliant-blue,
	urlcolor = magenta,
	plainpages = false
}

\setcounter{section}{-1}
\linespread{1.1}

% vertical space for tabular before/after \hline
\newcommand\Tstrut{\rule{0pt}{3.0ex}}         % = `top' strut
\newcommand\Bstrut{\rule[-1.5ex]{0pt}{0pt}}   % = `bottom' strut
\newcommand\grad{\mathbf{grad}\,}
\newcommand{\bx}{\bm{x}}
\newcommand{\rd}{\mathrm{d}}
\newcommand{\dx}{\rd\bx}
\newcommand{\bA}{\mathbf{A}}
\newcommand{\ba}{\bm{a}}
\newcommand{\inangle}[1]{\langle #1 \rangle}

\def\Plus{\texttt{+}}
\def\Minus{\texttt{-}}


